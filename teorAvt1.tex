%!TEX TS-program = xelatex



\documentclass[a4paper,14pt]{article}

\input{data/preambular.tex}
\begin{document} % конец преамбулы, начало документа
\begin{titlepage}
	\begin{center}
		Правительство Российской Федерации
	\end{center}

	\begin{center}
	Федеральное государственное автономное образовательное учреждение
	Высшего профессионального образования
	
	\end{center}
	
	\begin{center}
		\textbf{Национальный исследовательский университет}
		
		\textbf{«Высшая школа экономики»}
	\end{center}
	%\vspace{1ex}	
	\begin{center}
		МИЭМ
	\end{center}
	\begin{center}
		Департамент электронной инженерии
	\end{center}
	\vspace{1ex}
	\begin{center}
		\textbf{ОТЧЁТ}
		
		По домашней работе
		
		«Проектирование многоразрядного десятичного сумматоракомбинционного типа»
	\end{center}	
	
	\vspace{10ex}
	\begin{flushright}
		\textbf{Студент} группы БИВ-172
		
		Солодянкин Андрей Александрович		
	\end{flushright}
	\vspace{3ex}
	\begin{flushright}
		\textbf{Руководитель} 
		
		Бирюков И.И.

			
	\end{flushright}
	\vfill
	\begin{center}
		Москва \the\year г.
	\end{center}
\end{titlepage}

\section{Исходные данные для проектирования}


\begin{enumerate}
	\item Количество десятичных разрядов: $3$;
	\item Двоично-десятичный код, в котором находятся числа: $2421$;
	\item Система логических элементов: И-НЕ, И;
	\item Критерий оптимальности элементов для проектирования логических схем: минимальное число логических элементов (ЛЭ) в проектируемых схемах;
	\item Тип триггера для проектирования схемы управления синхронный D-треггер;
	\item Временные параметры синхронизирующей серии импульсов логических элементов: 
	время задержки в любом ЛЭ: 1 нс; 
	имульсы синхронизации длительностью 2 нс со скважностью 1. 
\end{enumerate}

\section{Разработка алгоритма выполнения арифметических операций сложения и вычитания многоразрядных чисел в заданом двоично-десятичном коде}

\begin{table}[H]
	\begin{tabular}{|c|c|}
		\hline
		\multicolumn{1}{|l|}{Цифра} & \multicolumn{1}{l|}{код (2421)} \\ \hline
		0 & 0000 \\ \hline
		1 & 0001 \\ \hline
		2 & 0010 \\ \hline
		3 & 0011 \\ \hline
		4 & 0100 \\ \hline
		5 & 1011 \\ \hline
		6 & 1100 \\ \hline
		7 & 1101 \\ \hline
		8 & 1110 \\ \hline
		9 & 1111 \\ \hline
	\end{tabular}
\end{table}

\subsection{Разработка алгоритма для одноразрядных десятичных чисел, получение величины коррекции и критерии ее ввода}

\begin{table}[H]
	\begin{tabular}{|c|c|c|c|c|c|c|c|c|c|c|}
		\hline
		& 0000 & 0001 & 0010 & 0011 & 0100 & 1011 & 1100 & 1101 & 1110 & 1111 \\ \hline
		0000 & \begin{tabular}[c]{@{}c@{}}00000\\ 00000\\ --\end{tabular} & \begin{tabular}[c]{@{}c@{}}0001\\ 0001\\ --\end{tabular} & \begin{tabular}[c]{@{}c@{}}0010\\ 0010\\ --\end{tabular} & \begin{tabular}[c]{@{}c@{}}0011\\ 0011\\ --\end{tabular} & \begin{tabular}[c]{@{}c@{}}0100\\ 0100\\ --\end{tabular} & \begin{tabular}[c]{@{}c@{}}1011\\ 1011\\ --\end{tabular} & \begin{tabular}[c]{@{}c@{}}1100\\ 1100\\ --\end{tabular} & \begin{tabular}[c]{@{}c@{}}1101\\ 1101\\ --\end{tabular} & \begin{tabular}[c]{@{}c@{}}1110\\ 1110\\ --\end{tabular} & \begin{tabular}[c]{@{}c@{}}1111\\ 1111\\ --\end{tabular} \\ \hline
		0001 & \begin{tabular}[c]{@{}c@{}}0001\\ 0001\\ --\end{tabular} & \begin{tabular}[c]{@{}c@{}}0010\\ 0010\\ --\end{tabular} & \begin{tabular}[c]{@{}c@{}}0011\\ 0011\\ --\end{tabular} & \begin{tabular}[c]{@{}c@{}}0100\\ 0100\\ --\end{tabular} & \begin{tabular}[c]{@{}c@{}}0101\\ 1011\\ 0110\end{tabular} & \begin{tabular}[c]{@{}c@{}}1100\\ 1100\\ --\end{tabular} & \begin{tabular}[c]{@{}c@{}}1101\\ 1101\\ --\end{tabular} & \begin{tabular}[c]{@{}c@{}}1110\\ 1110\\ --\end{tabular} & \begin{tabular}[c]{@{}c@{}}1111\\ 1111\\ --\end{tabular} & \begin{tabular}[c]{@{}c@{}}0000\\ 0000\\ --\end{tabular} \\ \hline
		0010 & \begin{tabular}[c]{@{}c@{}}0010\\ 0010\\ --\end{tabular} & \begin{tabular}[c]{@{}c@{}}0011\\ 0011\\ --\end{tabular} & \begin{tabular}[c]{@{}c@{}}0100\\ 0100\\ --\end{tabular} & \begin{tabular}[c]{@{}c@{}}0101\\ 1011\\ 0110\end{tabular} & \begin{tabular}[c]{@{}c@{}}0110\\ 1100\\ 0110\end{tabular} & \begin{tabular}[c]{@{}c@{}}1101\\ 1101\\ --\end{tabular} & \begin{tabular}[c]{@{}c@{}}1110\\ 1110\\ --\end{tabular} & \begin{tabular}[c]{@{}c@{}}1111\\ 1111\\ --\end{tabular} & \begin{tabular}[c]{@{}c@{}}0000\\ 0000\\ --\end{tabular} & \begin{tabular}[c]{@{}c@{}}0001\\ 0001\\ --\end{tabular} \\ \hline
		0011 & \begin{tabular}[c]{@{}c@{}}0011\\ 0011\\ --\end{tabular} & \begin{tabular}[c]{@{}c@{}}0100\\ 0100\\ --\end{tabular} & \begin{tabular}[c]{@{}c@{}}0101\\ 1011\\ 0110\end{tabular} & \begin{tabular}[c]{@{}c@{}}0110\\ 1100\\ 0110\end{tabular} & \begin{tabular}[c]{@{}c@{}}0111\\ 1101\\ 0110\end{tabular} & \begin{tabular}[c]{@{}c@{}}1110\\ 1110\\ --\end{tabular} & \begin{tabular}[c]{@{}c@{}}1111\\ 1111\\ --\end{tabular} & \begin{tabular}[c]{@{}c@{}}0000\\ 0000\\ --\end{tabular} & \begin{tabular}[c]{@{}c@{}}0001\\ 0001\\ --\end{tabular} & \begin{tabular}[c]{@{}c@{}}0010\\ 0010\\ --\end{tabular} \\ \hline
		0100 & \begin{tabular}[c]{@{}c@{}}0100\\ 0100\\ --\end{tabular} & \begin{tabular}[c]{@{}c@{}}0101\\ 1011\\ 0110\end{tabular} & \begin{tabular}[c]{@{}c@{}}0110\\ 1100\\ 0110\end{tabular} & \begin{tabular}[c]{@{}c@{}}0111\\ 1101\\ 0110\end{tabular} & \begin{tabular}[c]{@{}c@{}}1000\\ 1110\\ 0110\end{tabular} & \begin{tabular}[c]{@{}c@{}}1111\\ 1111\\ --\end{tabular} & \begin{tabular}[c]{@{}c@{}}0000\\ 0000\\ --\end{tabular} & \begin{tabular}[c]{@{}c@{}}0001\\ 0001\\ --\end{tabular} & \begin{tabular}[c]{@{}c@{}}0010\\ 0010\\ --\end{tabular} & \begin{tabular}[c]{@{}c@{}}0011\\ 0011\\ --\end{tabular} \\ \hline
		1011 & \begin{tabular}[c]{@{}c@{}}1011\\ 1011\\ --\end{tabular} & \begin{tabular}[c]{@{}c@{}}1100\\ 1100\\ --\end{tabular} & \begin{tabular}[c]{@{}c@{}}1101\\ 1101\\ --\end{tabular} & \begin{tabular}[c]{@{}c@{}}1110\\ 1110\\ --\end{tabular} & \begin{tabular}[c]{@{}c@{}}1111\\ 1111\\ --\end{tabular} & \begin{tabular}[c]{@{}c@{}}0110\\ 0000\\ 1010\end{tabular} & \begin{tabular}[c]{@{}c@{}}0111\\ 0001\\ 1010\end{tabular} & \begin{tabular}[c]{@{}c@{}}1000\\ 0010\\ 1010\end{tabular} & \begin{tabular}[c]{@{}c@{}}1001\\ 0011\\ 1010\end{tabular} & \begin{tabular}[c]{@{}c@{}}1010\\ 0100\\ 1010\end{tabular} \\ \hline
		1100 & \begin{tabular}[c]{@{}c@{}}1100\\ 1100\\ --\end{tabular} & \begin{tabular}[c]{@{}c@{}}1101\\ 1101\\ --\end{tabular} & \begin{tabular}[c]{@{}c@{}}1110\\ 1110\\ --\end{tabular} & \begin{tabular}[c]{@{}c@{}}1111\\ 1111\\ --\end{tabular} & \begin{tabular}[c]{@{}c@{}}0000\\ 0000\\ --\end{tabular} & \begin{tabular}[c]{@{}c@{}}0111\\ 0001\\ 1010\end{tabular} & \begin{tabular}[c]{@{}c@{}}1000\\ 0010\\ 1010\end{tabular} & \begin{tabular}[c]{@{}c@{}}1001\\ 0011\\ 1010\end{tabular} & \begin{tabular}[c]{@{}c@{}}1010\\ 0100\\ 1010\end{tabular} & \begin{tabular}[c]{@{}c@{}}1011\\ 1011\\ --\end{tabular} \\ \hline
		1101 & \begin{tabular}[c]{@{}c@{}}1101\\ 1101\\ --\end{tabular} & \begin{tabular}[c]{@{}c@{}}1110\\ 1110\\ --\end{tabular} & \begin{tabular}[c]{@{}c@{}}1111\\ 1111\\ --\end{tabular} & \begin{tabular}[c]{@{}c@{}}0000\\ 0000\\ --\end{tabular} & \begin{tabular}[c]{@{}c@{}}0001\\ 0001\\ --\end{tabular} & \begin{tabular}[c]{@{}c@{}}1000\\ 0010\\ 1010\end{tabular} & \begin{tabular}[c]{@{}c@{}}1001\\ 0011\\ 1010\end{tabular} & \begin{tabular}[c]{@{}c@{}}1010\\ 0100\\ 1010\end{tabular} & \begin{tabular}[c]{@{}c@{}}1011\\ 1011\\ --\end{tabular} & \begin{tabular}[c]{@{}c@{}}1100\\ 1100\\ --\end{tabular} \\ \hline
		1110 & \begin{tabular}[c]{@{}c@{}}1110\\ 1110\\ --\end{tabular} & \begin{tabular}[c]{@{}c@{}}1111\\ 1111\\ --\end{tabular} & \begin{tabular}[c]{@{}c@{}}0000\\ 0000\\ --\end{tabular} & \begin{tabular}[c]{@{}c@{}}0001\\ 0001\\ --\end{tabular} & \begin{tabular}[c]{@{}c@{}}0010\\ 0010\\ --\end{tabular} & \begin{tabular}[c]{@{}c@{}}1001\\ 0011\\ 1010\end{tabular} & \begin{tabular}[c]{@{}c@{}}1010\\ 0100\\ 1010\end{tabular} & \begin{tabular}[c]{@{}c@{}}1011\\ 1011\\ --\end{tabular} & \begin{tabular}[c]{@{}c@{}}1100\\ 1100\\ --\end{tabular} & \begin{tabular}[c]{@{}c@{}}1101\\ 1101\\ --\end{tabular} \\ \hline
		1111 & \begin{tabular}[c]{@{}c@{}}1111\\ 1111\\ --\end{tabular} & \begin{tabular}[c]{@{}c@{}}0000\\ 0000\\ --\end{tabular} & \begin{tabular}[c]{@{}c@{}}0001\\ 0001\\ --\end{tabular} & \begin{tabular}[c]{@{}c@{}}0010\\ 0010\\ --\end{tabular} & \begin{tabular}[c]{@{}c@{}}0011\\ 0011\\ --\end{tabular} & \begin{tabular}[c]{@{}c@{}}1010\\ 0100\\ 1010\end{tabular} & \begin{tabular}[c]{@{}c@{}}1011\\ 1011\\ --\end{tabular} & \begin{tabular}[c]{@{}c@{}}1100\\ 1100\\ --\end{tabular} & \begin{tabular}[c]{@{}c@{}}1101\\ 1101\\ --\end{tabular} & \begin{tabular}[c]{@{}c@{}}1110\\ 1110\\ --\end{tabular} \\ \hline
	\end{tabular}
\end{table}

Критерии ввода корректировки:

\begin{itemize}
	\item Если получена разрешенная комбинация и вне зависимости от наличия единицы переноса, корректировка не вводится, при этом единица переноса сохраняетсяж
	\item Если получена запрещенная комбинация и нет единицы переноса, то вводится корректировка $0110$;
	\item Если получена запрещенная комбинация и есть единица переноса, то вводится корректировка $1010$, при этом единица переноса сохраняется;
\end{itemize}

\subsection{Обобщение полученного алгоритма на многоразрядные числа при выполнении операции сложения и вычитания}

\subsection{Приведение шести примеров на следующие случаи сложения}

%Положительная величина (+A) складывается с другой положительной величиной (+B) с получением положительного результата (+C) без переполнения 
\subsubsection{(+A)+(+B)=(+C)}

%\begin{figure}[H]
%	\centering
%	\includegraphics[width=0.2\linewidth]{images/2_3_1_01}
%	\caption{}
%	\label{fig:23101}
%\end{figure}

\begin{figure}[H]
	\centering
	\includegraphics[width=0.7\linewidth]{primeri/screenshot001}
	\caption{}
	\label{fig:screenshot001}
\end{figure}



\subsubsection{(+A)+(-B)=(+C)}

%\begin{figure}[H]
%	\centering
%	\includegraphics[width=0.4\linewidth]{images/2_3_2_01}
%	\caption{}
%	\label{fig:23201}
%\end{figure}

\begin{figure}[H]
	\centering
	\includegraphics[width=0.7\linewidth]{primeri/screenshot002}
	\caption{}
	\label{fig:screenshot002}
\end{figure}


\subsubsection{(+A)+(-B)=(-C)}

%\begin{figure}[H]
%	\centering
%	\includegraphics[width=0.4\linewidth]{images/2_3_3_01}
%	\caption{}
%	\label{fig:23301}
%\end{figure}

\begin{figure}[H]
	\centering
	\includegraphics[width=0.7\linewidth]{primeri/screenshot003}
	\caption{}
	\label{fig:screenshot003}
\end{figure}

\subsubsection{(-A)+(-B)=(-C)}

%\begin{figure}[H]
%	\centering
%	\includegraphics[width=0.4\linewidth]{images/2_3_4_01}
%	\caption{}
%	\label{fig:23401}
%\end{figure}

\begin{figure}[H]
	\centering
	\includegraphics[width=0.7\linewidth]{primeri/screenshot004}
	\caption{}
	\label{fig:screenshot004}
\end{figure}

\subsubsection{(+A)+(+B)=(-C) — Переполнение разрядной сетки}

%\begin{figure}[H]
%	\centering
%	\includegraphics[width=0.4\linewidth]{images/2_3_5_01}
%	\caption{}
%	\label{fig:23501}
%\end{figure}

\begin{figure}[H]
	\centering
	\includegraphics[width=0.7\linewidth]{primeri/screenshot005}
	\caption{}
	\label{fig:screenshot005}
\end{figure}


\subsubsection{(-A)+(-B)=(+C) — Переполнение разрядной сетки}

%\begin{figure}[H]
%	\centering
%	\includegraphics[width=0.4\linewidth]{images/2_3_6_01}
%	\caption{}
%	\label{fig:23601}
%\end{figure}

\begin{figure}[H]
	\centering
	\includegraphics[width=0.7\linewidth]{primeri/screenshot006}
	\caption{}
	\label{fig:screenshot006}
\end{figure}

\section{Разработки функциональной схемы одноразрядного десятичного сумматора комбинационного типа}

\subsection{Разработка оптимальной схемы (с точки зрения критерия оптимальности) одноразрядного двоичного сумматора с учетом заданного базиса логических элементов}

Разработаем схему одноразрядного двоичного сумматора в базисе И, И-НЕ.

\begin{figure}[H]
	\centering
	\includegraphics[width=0.4\linewidth]{images/dvSum_el}
	\caption{одноразрядный двоичный сумматор}
	\label{fig:dvSum_el}
\end{figure}

$a$ -- перовое слагаемое;

$b$ -- второе слагаемое;

$c$ -- перенос из соседнего младшего разряда;

$S$ -- сумма в данном разряде;

$P$ -- перенос в соседний старший разряд.

В дальнейшем одноразрядный двоичный сумматор будет обозначаться как на рисунке \ref{fig:dvSum_el}.

\begin{table}[H]
\begin{center}
	\caption{\label{tab:dvSum} Таблица истинности для функций $S$ и $P$ суммы и перенса в однороразрядном двоичном сумматоре}
	\begin{tabular}{|l|l|l|l|l|}
		\hline
		$a$ & $b$ & $c$ & $S$ & $P$ \\ \hline
		0 & 0 & 0 & 0 & 0 \\ \hline
		0 & 0 & 1 & 1 & 0 \\ \hline
		0 & 1 & 0 & 1 & 0 \\ \hline
		0 & 1 & 1 & 0 & 1 \\ \hline
		1 & 0 & 0 & 1 & 0 \\ \hline
		1 & 0 & 1 & 0 & 1 \\ \hline
		1 & 1 & 0 & 0 & 1 \\ \hline
		1 & 1 & 1 & 1 & 1 \\ \hline
	\end{tabular}
\end{center}
\end{table}

\begin{table}[H]
	\begin{center}
		\caption{\label{tab:SDvSum} Диаграмма Вейча для функции $S$}
	\begin{tabular}{ccccc}
		& \multicolumn{2}{c}{$a$}                           & \multicolumn{2}{c}{$\overline{a}$}                          \\ \cline{2-5} 
		\multicolumn{1}{c|}{$b$}  & \multicolumn{1}{c|}{}  & \multicolumn{1}{c|}{1} & \multicolumn{1}{c|}{}  & \multicolumn{1}{c|}{1} \\ \cline{2-5} 
		\multicolumn{1}{c|}{$\overline{b}$} & \multicolumn{1}{c|}{1} & \multicolumn{1}{c|}{}  & \multicolumn{1}{c|}{1} & \multicolumn{1}{c|}{}  \\ \cline{2-5} 
		& $c$                     & \multicolumn{2}{c}{$\overline{c}$}                          & $c$                     
	\end{tabular}
\end{center}
\end{table}

Для создания логической схемы в базисе И, И-НЕ необходима диаграмма Вейча для функции $\overline{S}$.

\begin{table}[H]
	\begin{center}
		\caption{\label{tab:NSDvSum} Диаграмма Вейча для функции $\overline{S}$}
		\begin{tabular}{ccccc}
			& \multicolumn{2}{c}{$a$}                           & \multicolumn{2}{c}{$\overline{a}$}                          \\ \cline{2-5} 
			\multicolumn{1}{c|}{$b$}  & \multicolumn{1}{c|}{1}  & \multicolumn{1}{c|}{} & \multicolumn{1}{c|}{1}  & \multicolumn{1}{c|}{} \\ \cline{2-5} 
			\multicolumn{1}{c|}{$\overline{b}$} & \multicolumn{1}{c|}{} & \multicolumn{1}{c|}{1}  & \multicolumn{1}{c|}{} & \multicolumn{1}{c|}{1}  \\ \cline{2-5} 
			& $c$                     & \multicolumn{2}{c}{$\overline{c}$}                          & $c$                     
		\end{tabular}
	\end{center}
\end{table}

$\overline{S} = ab\bar{c} + a\bar{b}c + \bar{a}bc + \bar{a}\bar{b}\bar{c}$

$S = \overline{ab\bar{c} + a\bar{b}c + \bar{a}bc + \bar{a}\bar{b}\bar{c}}$

$$S = \overline{ab\bar{c}} * \overline{a\bar{b}c} * \overline{\bar{a}bc} * \overline{\bar{a}\bar{b}\bar{c}}$$

 В дальнейшем будем сразу строить диаграммы Вйеча для обратных функций.
 
 \begin{table}[H]
 	\begin{center}
 		\caption{\label{tab:NPDvSum} Диаграмма Вейча для функции $\overline{P}$}
 		\begin{tabular}{ccccc}
 			& \multicolumn{2}{c}{$a$}                           & \multicolumn{2}{c}{$\overline{a}$}                          \\ \cline{2-5} 
 			\multicolumn{1}{c|}{$b$}  & \multicolumn{1}{c|}{}  & \multicolumn{1}{c|}{} & \multicolumn{1}{c|}{}  & \multicolumn{1}{c|}{1} \\ \cline{2-5} 
 			\multicolumn{1}{c|}{$\overline{b}$} & \multicolumn{1}{c|}{1} & \multicolumn{1}{c|}{}  & \multicolumn{1}{c|}{1} & \multicolumn{1}{c|}{1}  \\ \cline{2-5} 
 			& $c$                     & \multicolumn{2}{c}{$\overline{c}$}                          & $c$                     
 		\end{tabular}
 	\end{center}
 \end{table}

$\overline{P} = \bar{b}\bar{c} + \bar{a}\bar{b} + \bar{a}\bar{c}$

$P = \overline{ \bar{b}\bar{c} + \bar{a}\bar{b} + \bar{a}\bar{c}}$

$$ P = \overline{ \bar{b}\bar{c}} *  \overline{ \bar{a}\bar{b}} * \overline{ \bar{a}\bar{c}}$$

\begin{figure}[H]
	\centering
	\includegraphics[width=0.6\linewidth]{images/dvSum_sh}
	\caption{Логическая схема одноразрядного двоичного сумматора}
	\label{fig:dvSum_sh}
\end{figure}

\subsection{Разработка схемы коррекции}

Разработаем корректор.

\begin{figure}[H]
	\centering
	\includegraphics[width=0.4\linewidth]{images/korr_el}
	\caption{корректор}
	\label{fig:korr_el}
\end{figure}

$\gamma_4, \gamma_3, \gamma_2, \gamma_1$ -- тетрада до корекции;

$\Pi_i$ -- перенос в следующую тетраду.

\begin{table}[H]
	\begin{center}
		\caption{\label{tab:dvKorr} Таблица истинности для функций $K_1$ и $K_2$ в корректоре}
	\begin{tabular}{|l|l|l|l|l|l|l|}
		\hline
		y4 & y3 & y2 & y1 & П & K1 & K2 \\ \hline
		0  & 0  & 0  & 0  & 0 & 0  & 0  \\ \hline
		0  & 0  & 0  & 0  & 1 & 0  & 0  \\ \hline
		0  & 0  & 0  & 1  & 0 & 0  & 0  \\ \hline
		0  & 0  & 0  & 1  & 1 & 0  & 0  \\ \hline
		0  & 0  & 1  & 0  & 0 & 0  & 0  \\ \hline
		0  & 0  & 1  & 0  & 1 & 0  & 0  \\ \hline
		0  & 0  & 1  & 1  & 0 & 0  & 0  \\ \hline
		0  & 0  & 1  & 1  & 1 & 0  & 0  \\ \hline
		0  & 1  & 0  & 0  & 0 & 0  & 0  \\ \hline
		0  & 1  & 0  & 0  & 1 & 0  & 0  \\ \hline
		0  & 1  & 0  & 1  & 0 & 1  & 0  \\ \hline
		0  & 1  & 0  & 1  & 1 & x  & x  \\ \hline
		0  & 1  & 1  & 0  & 0 & 1  & 0  \\ \hline
		0  & 1  & 1  & 0  & 1 & 0  & 1  \\ \hline
		0  & 1  & 1  & 1  & 0 & 1  & 0  \\ \hline
		0  & 1  & 1  & 1  & 1 & 0  & 1  \\ \hline
		1  & 0  & 0  & 0  & 0 & 1  & 0  \\ \hline
		1  & 0  & 0  & 0  & 1 & 0  & 1  \\ \hline
		1  & 0  & 0  & 1  & 0 & x  & x  \\ \hline
		1  & 0  & 0  & 1  & 1 & 0  & 1  \\ \hline
		1  & 0  & 1  & 0  & 0 & x  & x  \\ \hline
		1  & 0  & 1  & 0  & 1 & 0  & 1  \\ \hline
		1  & 0  & 1  & 1  & 0 & 0  & 0  \\ \hline
		1  & 0  & 1  & 1  & 1 & 0  & 0  \\ \hline
		1  & 1  & 0  & 0  & 0 & 0  & 0  \\ \hline
		1  & 1  & 0  & 0  & 1 & 0  & 0  \\ \hline
		1  & 1  & 0  & 1  & 0 & 0  & 0  \\ \hline
		1  & 1  & 0  & 1  & 1 & 0  & 0  \\ \hline
		1  & 1  & 1  & 0  & 0 & 0  & 0  \\ \hline
		1  & 1  & 1  & 0  & 1 & 0  & 0  \\ \hline
		1  & 1  & 1  & 1  & 0 & 0  & 0  \\ \hline
		1  & 1  & 1  & 1  & 1 & 0  & 0  \\ \hline
	\end{tabular}
\end{center}
\end{table}

\begin{table}[H]
	\begin{center}
		\caption{\label{tab:dvKorrK1} Диаграмма Вейча для функции $K_1$}
	\begin{tabular}{cccccccccc}
		& \multicolumn{4}{c}{$\Pi$}                                                                          & \multicolumn{4}{c}{$\overline{\Pi}$}                                                                            &                     \\
		& \multicolumn{2}{c}{$\gamma_4$}                        & \multicolumn{2}{c}{$\overline{\gamma_4}$}                        & \multicolumn{2}{c}{$\gamma_4$}                          & \multicolumn{2}{c}{$\overline{\gamma_4}$}                         &                     \\ \cline{2-9}
		\multicolumn{1}{c|}{\multirow{2}{*}{$\gamma_3$}}  & \multicolumn{1}{c|}{} & \multicolumn{1}{c|}{} & \multicolumn{1}{c|}{} & \multicolumn{1}{c|}{}  & \multicolumn{1}{c|}{}  & \multicolumn{1}{c|}{}  & \multicolumn{1}{c|}{1} & \multicolumn{1}{c|}{}  & $\overline{\gamma_1}$                 \\ \cline{2-9}
		\multicolumn{1}{c|}{}                     & \multicolumn{1}{c|}{} & \multicolumn{1}{c|}{} & \multicolumn{1}{c|}{} & \multicolumn{1}{c|}{x} & \multicolumn{1}{c|}{}  & \multicolumn{1}{c|}{}  & \multicolumn{1}{c|}{1} & \multicolumn{1}{c|}{1} & \multirow{2}{*}{$\gamma_1$} \\ \cline{2-9}
		\multicolumn{1}{c|}{\multirow{2}{*}{$\overline{\gamma_3}$}} & \multicolumn{1}{c|}{} & \multicolumn{1}{c|}{} & \multicolumn{1}{c|}{} & \multicolumn{1}{c|}{}  & \multicolumn{1}{c|}{x} & \multicolumn{1}{c|}{}  & \multicolumn{1}{c|}{}  & \multicolumn{1}{c|}{}  &                     \\ \cline{2-9}
		\multicolumn{1}{c|}{}                     & \multicolumn{1}{c|}{} & \multicolumn{1}{c|}{} & \multicolumn{1}{c|}{} & \multicolumn{1}{c|}{}  & \multicolumn{1}{c|}{1} & \multicolumn{1}{c|}{x} & \multicolumn{1}{c|}{}  & \multicolumn{1}{c|}{}  & $\overline{\gamma_1}$                 \\ \cline{2-9}
		& $\overline{\gamma_2}$                   & \multicolumn{2}{c}{$\gamma_2$}                        & \multicolumn{2}{c}{$\overline{\gamma_2}$}                         & \multicolumn{2}{c}{$\gamma_2$}                          & $\overline{\gamma_2}$                    &                    
	\end{tabular}
\end{center}
\end{table}

\begin{table}[H]
	\begin{center}
		\caption{\label{tab:dvKorrNK1} Диаграмма Вейча для функции $\overline{K_1}$}
	\begin{tabular}{cccccccccc}
		& \multicolumn{4}{c}{$\Pi$}                                                                             & \multicolumn{4}{c}{$\overline{\Pi}$}                                                                            &                     \\
		& \multicolumn{2}{c}{$\gamma_4$}                          & \multicolumn{2}{c}{$\overline{\gamma_4}$}                         & \multicolumn{2}{c}{$\gamma_4$}                          & \multicolumn{2}{c}{$\overline{\gamma_4}$}                         &                     \\ \cline{2-9}
		\multicolumn{1}{c|}{\multirow{2}{*}{$\gamma_3$}}  & \multicolumn{1}{c|}{1} & \multicolumn{1}{c|}{1} & \multicolumn{1}{c|}{1} & \multicolumn{1}{c|}{1} & \multicolumn{1}{c|}{1} & \multicolumn{1}{c|}{1} & \multicolumn{1}{c|}{}  & \multicolumn{1}{c|}{1} & $\overline{\gamma_1}$                 \\ \cline{2-9}
		\multicolumn{1}{c|}{}                     & \multicolumn{1}{c|}{1} & \multicolumn{1}{c|}{1} & \multicolumn{1}{c|}{1} & \multicolumn{1}{c|}{x} & \multicolumn{1}{c|}{1} & \multicolumn{1}{c|}{1} & \multicolumn{1}{c|}{}  & \multicolumn{1}{c|}{}  & \multirow{2}{*}{$\gamma_1$} \\ \cline{2-9}
		\multicolumn{1}{c|}{\multirow{2}{*}{$\overline{\gamma_3}$}} & \multicolumn{1}{c|}{1} & \multicolumn{1}{c|}{1} & \multicolumn{1}{c|}{1} & \multicolumn{1}{c|}{1} & \multicolumn{1}{c|}{x} & \multicolumn{1}{c|}{1} & \multicolumn{1}{c|}{1} & \multicolumn{1}{c|}{1} &                     \\ \cline{2-9}
		\multicolumn{1}{c|}{}                     & \multicolumn{1}{c|}{1} & \multicolumn{1}{c|}{1} & \multicolumn{1}{c|}{1} & \multicolumn{1}{c|}{1} & \multicolumn{1}{c|}{}  & \multicolumn{1}{c|}{x} & \multicolumn{1}{c|}{1} & \multicolumn{1}{c|}{1} & $\overline{\gamma_1}$                 \\ \cline{2-9}
		& $\overline{\gamma_2}$                    & \multicolumn{2}{c}{$\gamma_2$}                          & \multicolumn{2}{c}{$\overline{\gamma_2}$}                         & \multicolumn{2}{c}{$\gamma_2$}                          & $\overline{\gamma_2}$                    &                    
	\end{tabular}
\end{center}
\end{table}

$\overline{K_1} = \Pi + \gamma_3 \gamma_4 + \bar{\gamma_3} \bar{\gamma_4} + \gamma_1 \gamma_2 \bar{\gamma_3} + \bar{\gamma_1} \bar{\gamma_2} \gamma_3$

$$K_1 = \overline{\Pi} * \overline{\gamma_3 \gamma_4} * \overline{\bar{\gamma_3} \bar{\gamma_4}} * \overline{\gamma_1 \gamma_2 \bar{\gamma_3}} * \overline{\bar{\gamma_1} \bar{\gamma_2} \gamma_3}$$

\subsection{Разработка схемы одноразрядного десятичного сумматора}

\section{Разработка дополнительных схем для функционирования многоразрядного десятичного сумматора (все схемы проектируются в заданном базисе логических элементов с учетом критерия оптимальности)}
\subsection{Разработка преобразователя прямого кода в обратный для работы с отрицательными величинами}

\subsection{Разработка схемы, фиксирующей переполнение разрядной сетки}
\subsection{Разработка схемы для определения знака суммы}

\section{Разработки функциональной схемы многоразрядного десятичного сумматора}

\section{Разработка устройства управления для многоразрядного десятичного сумматора}
\subsection{Разработка входных и выходных регистров хранения числовой информации, участвующей в операции сложения}
\subsection{Разработка регистра признаков результата}
\subsection{Расчет временных параметров устройства управления}
\subsection{Разработка схемы для получения управляющих сигналов и схемы пуска выполнения операции сложения}

\section{Общая структура схемы многоразрядного десятичного сумматора комбинационного типа с устройством управления}

\section{Выводы по работе}

hjvjkj,hububylkuhuy
\end{document}